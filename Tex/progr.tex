\subsection{Разработка управляющей программы}

Основу прошивки Arduino Nano составляют добавленные библиотеки.

\begin{lstlisting}[frame=single]
#include <GyverStepper.h>
#include <EEPROM.h>
#include <PinChangeInterrupt.h>
#include <string.h>
#include <LiquidCrystal_I2C.h>
\end{lstlisting}

В дальнейшем, весь код можно разделить на части:

\begin{itemize}
	\item Объявление констант, классов, переменных и дефайнов;
	\item Основная работа;
	\item Функции/код связанные с экраном и его выводом;
	\item Функции связанные с записью количества монет в энергонезависимую EEPROM память;
	\item Обработка клафиш управления
\end{itemize}

Вся основная работа заключена в простой алгоритм: постоянно считывается данные с сенсора (фототранзистора), если они превысили определенный порог (сенсор заслонили от ИК), то у нас монетка и нужно отодвигать защёлку путём поворота шагового мотора. Для каждой монетки существует соответствующий угол, который напрямую связан с диаметром монеты. Зная на каком углу датчик вновь увидел ИК мы можем определить какая это монета.

\begin{lstlisting}[frame=single]
void loop {
	data = analogRead();
	if (data > BORDER) {
		if      (check(...)){this_coin = 0}
		else if (check(...)){this_coin = 1}
		...
	}
}
\end{lstlisting}

Из вышеприведённой вставки, главная функция имеет следующий функционал: поворачивается на определённый угол $R$, затем даёт время на возможный сброс монетки и срабатывание датчика $Y$, потом вновь считывает датчик и определяет -- упала ли монета или нет. 
\par\medskip

\begin{lstlisting}[frame=single]
bool check(int * mass, int R, int Y){
	int data = analogRead(inSEN);
	if (data > 100) {
		my_rotate(R);
	} else {
		comeback();
		return 1;
	}
	delay(Y);
	return 0;
}
\end{lstlisting}
Если монета была сброшена, то эта функция прорутит мотор до стопора, тем самым вернув защёлку в полностью закрытое состояние.
\par\medskip

По результату выхода из этой функции определяется тип монетки и он записывается в буффер имеющихся монет.
\par\medskip

По данным этого буффера на экран выводиться сумма и количество монет по номиналам.
\par\medskip