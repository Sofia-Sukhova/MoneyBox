\subsection{Сборка}

Основным этапом стала сборка итогового устройства.
\par\medskip

Электрическая схема была собрана из: шагового мотора, платы-драйвера шагового мотора, Arduino Nano, фототранзистора, ИК-светодиода, LCD экрана формата 2004 вместе с платой I2C, 2-х кнопок и провода USB (для питания). Схема соединений приведена ниже на рисунке~\ref{ris:scheme_electric}.

\begin{figure}[H]
	\centering
	\includegraphics[width=12cm]{pics/scheme_png.png}
	\caption{Схема соединения проводов}
	\label{ris:scheme_electric}
\end{figure}

При сборке были исправлены выявленные ошибки моделирования. В основном это касается отсутствующего отверстия для щели под ИК-светодиод и фототранзистор.
\par\medskip

В ходе процесса калибровки открытия щели для определенного номинала монет, была многократно доработана система задвижки -- методом перебора были отсеяны 2 наиболее подходящие пружины из порядка 20 оставшихся и одна из них была растянута/раскручена для достижения надёжной работы механизма.
\par\medskip

С ходе испытаний, шаговый мотор смог провернуть стопорное положение (упор в корпус) и тем самым деформировал стенку корпуса. Для восстановления работоспособности механизма задвижки корпус крышки был незначительно дефформирован -- была вставлена насквозь в стенку в месте упора небольшой штырёк, обеспечивающий надёжное стопорное положение для шагового мотора.
