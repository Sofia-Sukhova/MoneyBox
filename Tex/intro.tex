\section {Введение.}

На сегодняшний день большинство транзакций и расчётов происходит путём переводов из мобильных приложений, либо по бесконтактной оплате. Несмотря на рост безналичных платежей, мелкие монеты все еще остаются частью повседневных расчетов. Накоплениеи мелочи может мешать людям своим неопределённым положением -- мелочь лежит где-то на столе, в различных кармах. Организация копилки определит конкретное место для скопившейся мелочи и поможет вернуть её в оборот.
\par\medskip

Исходя из этих соображений, была выдвинута цель.

\section{Цель.}

Построить копилку с автоматическим определением номинала монет и накоплением общей суммы.

\subsection*{Задачи:}
\begin{itemize}
	\item Анализ существующих решений;
	\item Проектирование и изготовление устройства;
	\item Проверка и верификация надёжности работы
\end{itemize}

\section {Анализ существующих аналогов.}

Основными способами определять номинал монеты является определение одного из трёх характерных параметров: диаметр, толщина или магнитные свойства. 
\par\medskip

Определение номинала по магнитным свойствам возможно только с очень чувствительными (дорогими) датчиками.
\par\medskip

Определение номинала по толщине требует высокой точности -- до 0,01 мм. К тому же монета 50 копеек и 1 рубль имеют одинаковую толщину, что делает невозможным одновременный счёт в одном устройстве и копеек и рублей.
\par\medskip

Разница в диаметре различных монет составляет минимум 1 мм. Поэтому реализовать механизм определения номинала монеты целесообразнее именно по диаметру.
\par\medskip

Копилки со счетчиками монет не являются чем-то новым, они уже продаются в различных вариациях, однако такие устройства либо не являются автоматическими и требуют ручного ввода добавленной суммы (пример \cite{Ozon_manual_work}); либо их цена сильно завышена \cite{Ali_auto_work} (стоит отметить, что всевозможные конструкционные решения уже реализованны в широком классе устройств \cite{Cassida}, \cite{Sort_Cassida}, которые специализируются на крупном монетно/денежном обороте, что характерно для банковского сектора, предпринимателей).
\par\medskip

Помимо счётчиков, существуют ещё сортировщики монет, которые достаточно быстро разделяют монеты по номиналу на разные отсеки (примеры: \cite{Sort_Cassida},\cite{Sort_hill},\cite{Sort_default}). Все они имеют схожий механизм работы: монеты скатываются по определённой дорожке и разделяются в процессе движения. Любой подобный сортировщик можно модернизировать: добавить датчики на обнаружение проскакивания монеты на каждый из отсеков разделения и соединить все эти датчики с счётчиком монет -- получится необходимая нам копилка. К сожелению, все подобные сортировщики имеют большие размеры, что является недостатком -- копилка не должна занимать много места.
\par\medskip

Наиболее подходящим будет являтся пример устройста \cite{Gyver} -- диаметр определяется по интенсивности инфракрасного сигнала, приходящего на фоточувствительный элемент (фототранзистор); монеты перекрывают часть ИК-излучения и это изменение регистрирует датчик. Такая реализация допускает возможность ошибки за счет проскакивания монеты и неправильного считывания ее диаметра (к тому же в этой версии не предусмотрены удобные элементы управления).
\par\medskip

\section {Общее описание.}

Исходя из анализа существующих решений, можно составить требования к нашей копилке-счётчику: без сортировки, только автоматический счёт, высокая надёжность, низкая вероятность ошибки, компактность, удобство использования, хороший дизайн.
\par\medskip

Основу надёжного механизма будет составлять не просто щель, а щель с внутренней задвижкой -- такая конструкция не допустит проскакивания монеты без корректного считывания. К тому же, основным измерительным устройством в копилке будет сама щель с задвижкой -- отклонившись на определённую величину, через задвижку проскочат только те монеты, диаметр которых меньше этой величины. Расширяя щель поочерёдно от меньшего диаметра к большему, на каждом отколнии можно определить прошла ли монета или нет, и, в зависимости от итогового отклонения, определять какой был номинал у выпавшей монеты.
\par\medskip

За движение задвижки будет отвечать шаговый мотор. Тем самым, измерение с какого-либо датчика перекладывается на установление заданного угла шагового мотора. Фототранзистор и ИК-светодиодом также установлены в щель -- их задача определять, есть ли в щели монета. Именно фототранзистор и ИК-светодиод будут задавать начало измерения номинала монеты и постепенное открытие щели, а также поределять момент проскакивания монеты при открывании щели.
\par\medskip

Для удобства эксплуатации данные о принятых монетах выводятся на дисплей в двух режимах: общая сумма в копилке и распределение содержимого по номиналам. Управление копилкой осуществляется через 2 кнопки.
\par\medskip

Питание всей электроники осуществело от 5 вольт постоянного тока по кабелю USB.