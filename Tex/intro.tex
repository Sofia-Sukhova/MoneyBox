\section {Цель}
Построить копилку с автоматическим определением номинала монет и накоплением общей суммы.

\subsection*{Задачи}
\begin{itemize}
	\item Анализ существующих решений;
	\item Проектирование и изготовление устройства;
	\item Проверка и верификация надёжности работы
\end{itemize}

\section {Общее описание}

Копилка со счетчиком монет. В монетной щели расположена клиновидная задвижка, присоединенная к шаговому мотору и фототранзистору. С помощью фототранзистора фиксируется наличие/отсутствие монеты в щели, а ее номинал определяется по повороту шагового мотора.
\par\medskip

Данные о принятых монетах выводятся на дисплей в двух режимах: общая сумма в копилке и распределение содержимого по номиналам.

\section {Анализ существующих аналогов}

Копилки со счетчиками монет в различных вариациях представлены на маркетплейсах [\href{https://www.ozon.ru/product/yilijukj-kopilka-dlya-deneg-11h20-sm-1358489566/?asb=sCr1GmIlWB7Fjby%252BPYnWUNFeAjvdIw5ordlD0iRvbOM%253D&asb2=sn4m9gpYWZy_EEZH7OfMzDGoCOjFWmDehfvE83sshNuqVbbHLKnkaqdBme_UnwQL9zPGT5WB-Cli3HsA9FpOcg&avtc=1&avte=2&avts=1717862839&keywords=копилка+счетчик}{1}], [\href{https://aliexpress.ru/item/1005002928697666.html?sku_id=12000028995331515&spm=a2g2w.productlist.search_results.1.5d644f98Nvin2D}{2}], однако такие устройства либо не являются автоматическими, и требуют ручного ввода добавленной суммы; либо их цена сильно завышена[\href{https://www.ozon.ru/product/schetchik-i-sortirovshchik-monet-cassida-coinmax-1324388430/?asb=FCEZ8bYDzfLDlR1hbPXsPN7zg5D8DCyINZKx%252ByUJkgQ%253D&asb2=pFy0DP03AN1Um2JNqhlwAfVK8Tqf4HIHittgnkgxfxYFRn3HUd_mHS4O0_9rIrPfWd8S0KDwvHJyzVXS0PcbQw&avtc=1&avte=2&avts=1718018978&keywords=cassida+%D0%BC%D0%BE%D0%BD%D0%B5%D1%82%D1%8B}{3}].
\par\medskip

Определять номинал монеты можно по одному из трех признаков: диаметр, толщина или состав (магнитные свойства). Определение номинала по магнитным свойствам возможно только с очень чувствительными (дорогими) датчиками. Определение номинала по толщине требует высокой точности -- до 0,01 мм. Разница в диаметрах монет составляет минимум 1 мм. Поэтому реализовать механизм определения номинала монеты целесообразнее именно по диаметру.
\par\medskip

Пример устройства, реализующего такой механизм, выполнен AlexGyver [\href{https://www.youtube.com/watch?v=lH4qfGlK2Qk}{4}]. В данной идее диаметр определялся по интенсивности инфракрасного сигнала, приходящего на фоточувствительный элемент (фототранзистор).  Такая реализация допускает возможность ошибки за счет проскакивания монеты и неправильного считывания ее диаметра. К тому же в этой версии не предусмотрены удобные элементы управления (кнопки reset/переключение дисплея).
\par\medskip

Другими аналогичными устройствами являются многочисленные реализации сортировщиков монет, основанные на проскакивании монет через щели определенного диаметра [\href{https://www.youtube.com/watch?v=u0fAAhLuL24&ab_channel=KryzerChannel}{5}]. Такие устройства обладают высокой скоростью сортировки и счета, однако механизм занимает достаточно много места. Эта реализация хорошо подходит для промышленного использования (большого оборота монет), поэтому существует широкий сегмент подобных устройств, обладающих высокой надежностью и, соответственно, высокой ценой. 
\par\medskip

Таким образом, наше устройство выделяется на фоне существующих непромышленных аналогов автоматической и надежной работой, что делает его удобным в использовании.

