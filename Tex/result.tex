\section {Тестирование.}

Основными задачами тестирования были проверка надежности механизма и калибровка шагового мотора для корректного определения каждого номинала.
\par\medskip

Проверка включала в себя ручной пересчёт монет и автоматический с помощью копилки. Итоговый результат откалиброванной копилки-счётчика полностью совпал по всем номиналам с ручным счётом. В ходе тестирования было считано:
\begin{itemize}
	\item 12 монет по 10 рублей;
	\item 56 монет по 5 рублей;
	\item 97 монет по 2 рубля;
	\item 170 монет по 1 рублю;
	\item 31 монета 50 копеек;
	\item 34 монет 10 копеек;
	\item 5 монет по 5 копеек;
	\item 0 монет по 1 копейке
\end{itemize}

В процессе тестирования были выявлены и исправлены некоторые недостатки модели. Выявилось, что шаговый мотор может перейти стопорное положение (упор в корпус) и тем самым деформировать стенку корпуса. Также была многократно доработана система задвижки -- были выбраны 2 наиболее подходящие пружины из порядка 20, и одна из них была дополнительно деформирована для достижения надёжной работы механизма.

\section {Результаты.}

Было спроектировано, изготовлено и протестировано устройство — копилка со счетчиком монет. Функционально устройство работает исправно. 
\par\medskip

Документация и файлы прошивки, моделей CAD расположены на GitHub \cite{github}.
\par\medskip

В качестве дальнейшего развития можно доработать управляющую программу и внедрить в нее новые функции, а также доработать модель и исправить недостатки, выявленные при тестировании модели.